\section{Modeling}
\subsection{Introduction}
We model the system as N couples of transmitter and receiver which communicate through C Channels. A collision can occur on a channel if more than one transmitters want to transmit a packet in that channel. The transmitter stores in a queue the packets that it wants to transmit and, then, it sends them; the channels "knows" if a collision occurs and handle it; the receiver only receive packets.  
\subsection{General Assumptions}
The following general assumptions have been made:
\begin{itemize}
	\item \textbf{Slotted}: packets are attempted to be transmitter by the Transmitter only at the \textbf{beginning of the timeslot}
	\item \textbf{Constant Packet Size and Transmission Rate}: each packet has a constant packet size and each transmitter has a constant and equal transmission rate for which to transmit a packet (without collision) from the receiver to the transmitter will last \textbf{one timeslot}.
	\item \textbf{No Propagation Error in the channel}: the only cause of a failed transmission has to be considered as the packet collision. Other causes, (i.e. path-loss, shadowing, small-scale fading), are neglected. 
	\item \textbf{FIFO Queues of unlimited Capacity at the Transmitter} 
	\item \textbf{Transmitters and Receivers always synchronized with the timeslot period}: in addition the receiver knows in which channel the transmitter will try to send his packet in each timeslot and the receiver will be ready to listen in the correct channel.
	\item \textbf{After an eventual collision the packet will change his channel choice}
\end{itemize}

\subsection{Preliminar Validation}
Before the implementation a preliminar validation phase is necessary to ensure that the model is correct. Let analyze if the assumptions made in the previous section are reasonable:
\begin{itemize}
	\item The pure slotted assumption is reasonable due to the fact that exist some network protocols which work under this assumption, see for instance Slotted Aloha that is the most famous one.
	\item We consider the packet size constant because, in a network, there are small packet and huge packet, but if we want to consider the performance, then we have to take a mean length, otherwise it's possible that we consider too small or too large packets. Furthermore if the packet length is so large that more than one slots are needed, we can consider, from the viewpoint of the model, that this unique packet send in two different slots is like two packets of fixed-length send each one in a slot.
	\item The critical issue of every slotted network is the one related to the collisions: they have an huge impact on the general performance of the network, so it is reasonable to neglect the other propagation errors that are not network-specific or that depend from the environment (as path-loss).
	%\item Each memory has a limited capacity, so for sure the buffer of the transmitter can't be infinite. The access policy is the default one for the most queue systems (i.e. FIFO).
	\item It's reasonable that the transmitter and the receiver are synchronized with the timeslot period because otherwise it would be very difficult every type of communications between the two entities. Moreover also slotted ALOHA requires a synchronization of this type.
	\item When a packet collide it's reasonable to think that the transmitter will change the transmission channel in order to avoid another collision. Indeed this along with the back-off time are the techniques that should avoid another collision. 
\end{itemize}
\subsection{Factors}
The following factors have been defined which may affect the performance of the system:
\begin{itemize}
	\item \textbf{N}: Transmitter-Receiver Couples.
	\item \textbf{C}: numbers of Channels.
	\item \textbf{p}: probability of success for sending a packet in the current timeslot for a Transmitter.
	\item \textbf{$lambda$}: exponential distribution rate of packets arrival at the Transmitter.
	%\item \textbf{K}: Transmitter Buffer Size.
	\item $T_{deadline}$: deadline in case of communication with a real-time system.
	\item $T_{timeslot}$: timeslot duration.
	
\end{itemize}