\section{Modeling}
\subsection{Introduction}

\subsection{General Assumptions}
The following general assumptions have been made:
\begin{itemize}
	\item \textbf{Pure Slotted}: packets are attempted to be transmitter by the Transmitter only at the \textbf{beginning of the timeslot}
	\item \textbf{Constant Packet Size and Transmission Rate}: each packet has a constant packet size and each transmitter has a constant and equal transmission rate for which to transmit a packet (without collision) from the receiver to the transmitter will last \textbf{one timeslot}.
	\item \textbf{No Propagation Error in the channel}: the only cause of a failed transmission has to be considered as the packet collision. Other causes, i.e. path-loss, are 
	\item \textbf{FIFO Queues of limited Capacity at the Transmitter} 
	\item \textbf{Transmitters and Receivers always synchronized with the timeslot period}: in addition the receiver knows in which channel the transmitter will try to send his packet in each timeslot and the receiver will be ready to listen in the correct channel.
	\item \textbf{After an eventual collision the packet will change his channel choice}
\end{itemize}

\subsection{Preliminar Validation}

\subsection{Factors}
The following factors have been defined which may affect the performance of the system:
\begin{itemize}
	\item \textbf{N}: Transmitter-Receiver Couples.
	\item \textbf{C}: numbers of Channels.
	\item \textbf{p}: probability of success for sending a packet in the current timeslot for a Transmitter.
	\item \textbf{$lambda$}: exponential distribution rate of packets arrival at the Transmitter.
	\item \textbf{K}: Transmitter Buffer Size.
	\item $T_{deadline}$: deadline in case of communication with a real-time system.
	\item $T_{timeslot}$: timeslot duration. 
\end{itemize}