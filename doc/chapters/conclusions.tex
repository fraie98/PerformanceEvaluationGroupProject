% !TeX spellcheck = en_GB
\section{Conclusions}
The final consequences of the report are shown in this section. 
The results obtained in the \textbf{Response Time Explosion scenario} underlines \textbf{three aspects}:
\begin{itemize}
	\item The \textbf{No-Change of channel sub-scenario} (in case of collision), is \textbf{worst in terms of throughput} than the Change Channel one, \textbf{when the number of trasmitters start to grow} (so more collisions expected mantaining fixed the number of channels that will makes more clear the difference between the two scenarios)
	\item An \textbf{higher Send Probability is better} in this case because \textbf{maximizes the throughput} (consideration on Response Time can't be done). This increase start grows faster at the beginning of the increase of p. 
	\item The \textbf{Throughput keeps increasing with the number of Transmitters} (with our ranges) and is the \textbf{most relevant factor} as underlined in the relative $2^{k}r$ experiment.
\end{itemize} 
%Considering the join effect of the KPIs we can say that the Slotted Random-Access Wireless Network studied has good performance and the following observations about it can be done. 

\noindent %As first observation we can say that the system is fair. We can say this due to the Lorenz Curve (figure \ref{img: insight2_throughput}, \ref{img: insight2_respTime}, \ref{img: insight3_respTime}).

\noindent Moreover the overall performance of the system depends heavily from the mean inter-arrival time. In fact if it is lower than 125 ms no observations about the response time can be done and no meaningful data can be obtained because the response time diverges. Then if we want to use this type of network for application with mean inter-arrival time lower than 125 ms, this is discouraged at least for the situations that we have studied. Instead, if for example we want to use the protocol in a network used by a factory to link several machines (with the N that we have used in the experiment), and each machine transmits a packet each $x$ ms with $x\ge125$, then this network protocol can be suggested.

\noindent In this context, the results obtained in the conditions previously explained and shown, are acceptable and in particular we can infer the following:
\begin{itemize}
	\item The throughput maintains a good value for all values and it is greater when the traffic is higher as we expect (the more the transmitter the more the data and so the more the throughput).
	
	\item The response time is acceptable also in the worst case and it is good for the other cases, both in high traffic condition and in low traffic condition. In any case the performance in the worst case, even if can be acceptable, it is poor because the mean response time is comparable with the mean inter-arrival time and so for a low probability of transmission (so in a time slot we may have a lot of transmitters that don't transmit) the network doesn't perform well.
	\item The higher the mean inter-arrival time the lower the throughput and the lower the response time. A very low mean inter-arrival time can be a problem from the response time viewpoint.
	\item An high probability of transmission is always better (also for the response time explosion scenario).
\end{itemize}

\noindent Comparing the two scenarios we found we can also state that, for our ranges of interest, the \textbf{presence Bernoullian Experiment tends to decrease the performance} of our system (Throughput in the Response Time Scenario, Response time in the Limited Response time Scenario). So we can think to \textbf{remove it from the protocol to tune our system}.