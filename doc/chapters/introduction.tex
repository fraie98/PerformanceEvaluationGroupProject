\section{Introduction}
\subsection{Problem Description}
From the group project assignment:
\vspace{0.4cm}

\textit{In a \textbf{slotted random-access network}, \textbf{N couples transmitter-receiver} share the same communication
medium, which consists of \textbf{C separate channels}. Multiple attempts to use the same channel in the
same slot by different transmissions will lead to collision, hence no receiver listening on that
channel will be able to decode the message.
Assume that each of the N transmitters generate packets according to an \textbf{exponential interarrival
distribution}, and picks its channel at random on every new transmission. Before sending a packet, it
keeps extracting a value from a \textbf{Bernoullian RV with success probability p} on every slot, until it
achieves success. Then it transmits the packet and starts over. If a collision occurs, then the
transmitter backs off for a random number of slots (see later), and then starts over the whole
Bernoullian experiment.
The number of backoff slots is extracted as $U(1, 2^{x+1})$, where x is the number of collisions
experienced by the packet being transmitted.}
\subsection{Objectives}
The aim of the project report is the \textit{Assessment of the Effectiveness of the Slotted Random-Access Network Protocol} described in the latter paragraph.
\subsection{Performance Indexes}
In order to define a metric of performance of the objective, the following Performance Indexes are defined:
\begin{itemize}
	\item \textbf{Throughput}: let $Tp$ be the Throughput to be measured, $N_{p}$ the number of packets successfully sent in the same timeslot, $T_{timeslot}$ the timeslot duration, the Throughput can be measured as:
	
	\begin{equation}
	Tp = \frac{N_{packets  timeslot}}{T_{timeslot}} = [s^{-1}]
	\end{equation}
	
	\item \textbf{Response Time}: defined as the time that occurs from the first appearance of one packet at the Transmitter up to the reception of the packet at the Receiver.
	\item \textbf{Percentage of Loss Packets}: due to the fact that the transmitter has a limited buffer capacity (see next chapter), there is a need to consider even this variable as a performance index.
	\item \textbf{Network Traffic}: (?)
	\item \textbf{Percentage of Deadlines not respected}: if we consider running this type of communication protocol of a real-time system with his own deadlines, there is a need to consider even this variable as a performance index.
\end{itemize}